\documentclass[border=3.5cm]{standalone}

% Required packages
\usepackage{tikz}
\usetikzlibrary{arrows.meta, shapes, positioning, fit, backgrounds}

\usepackage{pgfplots}
\pgfplotsset{compat=1.18}

\usepackage[dvipsnames]{xcolor} % for colours

% set main font and disable separate font for math/symbols
\usepackage[no-math]{fontspec}
\setmainfont[Ligatures=TeX]{Inter}
\usepackage[defaultmathsizes]{mathastext}

\begin{document}

% define colours
\definecolor{soibAccent}{HTML}{952C12}
\definecolor{soibYellow}{HTML}{d4c64a}
\definecolor{soibGreen}{HTML}{a2bd5b}

\tikzset{
	thick,
	node distance=2cm,
	% styles
	TITLE/.style={font=\Huge, scale=3, fill=white, font=\bfseries},
	BLOCK.TITLE/.style={font=\Large, rectangle, ultra thick, align=center, inner sep=0.3cm},
	BLOCK/.style={BLOCK.TITLE, minimum width=15cm, draw=soibGreen, fill=soibGreen!40},
	STEP/.style={font=\large, rectangle, very thick, minimum width=7cm, minimum height=2cm, align=center, 
				draw=soibYellow, fill=soibYellow!40},
	INPUT.DATA/.style={font=\normalsize, align=right, inner sep=0.3cm},
	SUBSTEP/.style={font=\normalsize, align=left, anchor=west, rectangle, draw=black, 
				minimum width=7cm, minimum height=3cm, text width=6.5cm},
	SUBSTEP.MINI/.style={font=\normalsize, align=left, anchor=west, rectangle, draw=black, 
				minimum width=4.5cm, text width=4.3cm, inner sep=0.3cm},
	% arrows
	ARROW/.style={-{Stealth[length=0.6cm]}, ultra thick},	
	ARROW.MINI/.style={-{Stealth[length=0.3cm]}},	
	SUBSTEP.LINE/.style={draw=black!75, dashed, dash pattern=on 10pt off 3pt},
}


% trends

\begin{tikzpicture}

% titles

\node[TITLE, xshift=10cm]
	(title) {The SoIB Pipeline};


% BLOCK 1
 \node[BLOCK.TITLE] [anchor=north, below=of title, yshift=-4cm] 
    	(b1)	{1. Setting up data for analyses};
  
% Step 1
\node[STEP] [draw, below=of b1]
	(b1s1) 	{Readying raw data};
	% input data
	\node[INPUT.DATA, left=of b1s1, yshift=2cm, xshift=-4cm]
		(data1)	{eBird data};
	\node[INPUT.DATA, left=of b1s1, yshift=0cm, xshift=-4cm]
		(data2)	{Species attributes};
	\node[INPUT.DATA, left=of b1s1, yshift=-2cm, xshift=-4cm]
		(data3)	{Spatial polgyons\\and grids};
% Step 2
\node[STEP] [draw, below=of b1s1]
	(b1s2) 	{Reshaping eBird data};
% Step 3
\node[STEP] [draw, below=of b1s2]
	(b1s3) 	{Curating eBird data};
% Step 4
\node[STEP] [draw, below=of b1s3]
	(b1s4) 	{Selecting species and checklists};
	
% substeps
\node[SUBSTEP, draw, right=of b1s2.east, xshift=3cm]
	(b1ss2)	{Converting calendar year to migratory\\year\\
				\ \\
				Creating variables for analyses\\
				\ \\
				Removal of vagrant records};
\node[SUBSTEP, draw, right=of b1s3.east, xshift=3cm]
	(b1ss3)	{Removing observations unsuitable for public output\\
				\ \\
				Reviewing checklist completeness};
\node[SUBSTEP, draw, right=of b1s4.east, xshift=3cm]
	(b1ss4)	{Selecting species\\
				\ \\
				Constraining data for a species\\
				\ \\
				Reducing spatial non-independence of checklists};
				
% lines to substeps
\draw[SUBSTEP.LINE]
	(b1s2.north east)		edge		(b1ss2.north west)
	(b1s2.south east)		edge		(b1ss2.south west)
	(b1s3.north east)		edge		(b1ss3.north west)
	(b1s3.south east)		edge		(b1ss3.south west)
	(b1s4.north east)		edge		(b1ss4.north west)
	(b1s4.south east)		edge		(b1ss4.south west);
    	  	
\begin{pgfonlayer}{background}
\node[BLOCK][fit={(b1) (b1s1) (b1s2) (b1s3) (b1s4)}, inner sep=1cm]
	(b1box)	{};
\end{pgfonlayer}


% BLOCK 2
 \node[BLOCK.TITLE] [anchor=north, below=of b1box.south, yshift=-2cm, xshift=-10cm] 
    	(b2)	{2. Estimating Long-term Change (LTC)\\and Current Annual Trend (CAT)};
  
% Step 1
\node[STEP] [draw, below=of b2]
	(b2s1) 	{Model frequency of reporting};
% Step 2
\node[STEP] [draw, below=of b2s1]
	(b2s2) 	{Categorize LTC and CAT};
    	  	
\begin{pgfonlayer}{background}
\node[BLOCK][fit={(b2) (b2s1) (b2s2)}, inner sep=1cm] 
	(b2box)	{};
\end{pgfonlayer}


% substeps
\node[SUBSTEP.MINI, draw, left=of b2s1.west, xshift=-3cm]
	(b2ss1)	{Model confidence check\\
				\ \\
				Calculating LTC and CAT from predicted values\\
				\ \\
				Sensitivity check};
				
% lines to substeps
\draw[SUBSTEP.LINE]
	(b2s1.north west)		edge		(b2ss1.north east)
	(b2s1.south west)		edge		(b2ss1.south east);


% BLOCK 3
 \node[BLOCK.TITLE] [anchor=north, below=of b1box.south, yshift=-2cm, xshift=10cm] 
    	(b3)	{3. Estimating Distribution\\Range Size (DRS)};
  
% Step 1
\node[STEP] [draw, below=of b3]
	(b3s1) 	{Model occupancy};
% Step 2
\node[STEP] [draw, below=of b3s1]
	(b3s2) 	{Categorize DRS};
    	  	
\begin{pgfonlayer}{background}
\node[BLOCK][fit={(b3) (b3s1) (b3s2)}, inner sep=1cm] 
	(b3box)	{};
\end{pgfonlayer}


% substeps
\node[SUBSTEP.MINI, draw, right=of b3s1.east, xshift=3cm]
	(b3ss1)	{Calculating DRS from predicted values};
				
% lines to substeps
\draw[SUBSTEP.LINE]
	(b3s1.north east)		edge		(b3ss1.north west)
	(b3s1.south east)		edge		(b3ss1.south west);

% BLOCK 4

\node[BLOCK.TITLE][anchor=north, below=of b2box.south, yshift=-2cm, xshift=10cm] 
	(b4)	{4. SoIB Priority Status};
	
%\node[BLOCK][anchor=north, below left=of b3box, yshift=-2cm, inner sep=1cm] 
%	(b4box)	{4. SoIB Priority Status};
\begin{pgfonlayer}{background}
\node[BLOCK][fit={(b4)}, inner sep=1cm] 
	(b4box)	{};
\end{pgfonlayer}




% arrows

\draw[ARROW.MINI] 
	(data1) 	edge 	[out=-10, in=180]		(b1s1.west);
\draw[ARROW.MINI] 
	(data2) 	edge 					(b1s1.west);
\draw[ARROW.MINI] 
	(data3) 	edge 	[out=10, in=180]		(b1s1.west);
	
\draw[ARROW]
	(b1s1.south)	edge		(b1s2.north)
	(b1s2.south)	edge 		(b1s3.north)
	(b1s3.south)	to		(b1s4.north);
	
\draw[ARROW]
	(b1s4.south)	-- ++(0cm, -2cm) 	 	-|			(b2box.north);
\draw[ARROW]
	(b1s4.south)	-- ++(0cm, -2cm) 	 	-|			(b3box.north);	
	
\draw[ARROW]
	(b2s1.south)	to		(b2s2.north);
\draw[ARROW]
	(b2s2.south)	-- ++(0cm, -2cm) 	 	-|			(b4box.north);
\draw[ARROW]
	(b3s1.south)	to		(b3s2.north);
\draw[ARROW]
	(b3s2.south)	-- ++(0cm, -2cm) 	 	-|			(b4box.north);
	
\end{tikzpicture}

\end{document}
